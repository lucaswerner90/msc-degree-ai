\cleardoublepage

\chapter*{Resumen}
\label{resumen}
\addcontentsline{toc}{chapter}{Resumen}

El siguiente trabajo presenta una alternativa mediante el uso de técnicas de aprendizaje por refuerzo al algoritmo de seguimiento de una persona en tiempo real por parte de un dron. A diferencia de estudios ya realizados anteriormente y los cuales iremos detallando a lo largo del siguiente documento, no utilizaremos simuladores para la obtención de los datos, sino que utilizaremos la propia cámara del dispositivo para obtener estos, de manera que el entorno final sea lo más parecido al entorno de entrenamiento.
\medskip

Construiremos en primer lugar una base teórica, en la que explicaremos brevemente los conceptos que involucra el siguiente trabajo y que da pie a entender las diferentes decisiones tomadas durante su ejecución.
\medskip

Luego realizaremos un análisis de nuestros objetivos, tomaremos la definición de nuestro problema, y lo estructuraremos para adaptarlo al modelo de problema de aprendizaje por refuerzo. Además iremos comentando los problemas que cada uno de estos pasos conlleva y su evolución y distintas variantes a lo largo de los diferentes experimentos, incluyendo decisiones en el entrenamiento de nuestros agentes, así como la redefinición y adaptación mediante análisis de resultados, pero también mediante prueba y error de cada una de las piezas.
\medskip

Finalmente haremos un análisis de los algoritmos utilizados y trataremos los diferentes problemas que estos fueron ocasionando durante el transcurso del entrenamiento. 

