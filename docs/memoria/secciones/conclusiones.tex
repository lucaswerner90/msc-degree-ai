% CONCLUSIONES

\chapter{Conclusiones}
\label{conclusiones}

Tras haber analizado los diferentes experimentos, haremos un breve repaso de las conclusiones principales que obtuvimos:

\begin{enumerate}[label=\destacado{\arabic*.}]
  \setlength\itemsep{1em}
  \item \textbf{La recompensa juega un papel importante.}
  \medskip

  Como vimos durante los diferentes experimentos, y aunque solo hayamos comentado un subconjunto de los totales realizados, la función de recompensa jugó sin duda un papel clave en todo el trabajo. Las diferentes decisiones sobre esta intentaban apaciguar los diferentes problemas que iban surgiendo tras analizar que los resultados obtenidos no se asemejaban a nuestro idea inicial.
  \medskip

  \item \textbf{Los hiperparámetros no fueron decisivos.}
  \medskip

  Generalmente asociamos que la convergencia de nuestras redes neuronales está asociada con la elección de buenos hiperparámetros en el entrenamiento, tal y como aprendimos a lo largo del master. Sin embargo, en nuestro caso esto no fue un problema puesto que, si bien el agente podía llegar a un punto de recompensa media antes o después, esto no determinaba el resultado final. Hicimos pruebas con diferentes técnicas de inicialización de pesos, diferentes tasas de aprendizaje e incluímos técnicas conocidas como \textit{Dropout}, sin embargo el resultado no variaba de manera significativa.
  \medskip

  \item \textbf{Definir correctamente el problema y sus componentes.}
  
  \medskip
  Esta es la conclusión que finalmente podríamos destacar. Bajo nuestra experiencia a lo largo del desarrollo de este proyecto pudimos observar una mejoría y un empeoramiento del resultado de nuestro agente modificando levemente en algunos casos las definiciones de nuestro entorno, ya sea a través de un cambio en la función de recompensa o en el criterio de parada del entrenamiento. Lo cual nos indica la dificultad que tiene el aplicar algoritmos de aprendizaje por refuerzo a escenarios reales (recordando incluso que el problema que afrontamos contaba con limitaciones con las cuales un producto industrial real tendría que lidiar ).
  \medskip

\end{enumerate}

Para finalizar también nos gustaría dejar las conclusiones que obtuvimos a nivel personal tras la realización de este trabajo. Empezaremos diciendo que sin duda alguna el trabajo nos pareció interesante desde el primer momento, no solo porque queríamos aplicar toda la teoría aprendida durante el master, sino también porque era una rama a la cual no le pudimos dedicar todo el tiempo que hubiesemos querido en su momento. 
\medskip

A pesar de que bajo nuestro punto de vista el trabajo quedó lejos de ser como teníamos en mente desde un principio, debido a situaciones personales y a los diferentes problemas que comentamos en secciones anteriores: el entrenamiento de los agentes, la definición del problema y en general el intentar llevar al mundo real un tema que por lo general siempre se relaciona con entornos simulados, con todas las dificultades que ello conlleva, nos quedamos satisfechos de que fue una prueba de las capacidades de este tipo de aprendizajes (y nuestra también) y de que nos motiva aún más a seguir estudiando sobre el tema y que la experiencia aprendida durante estos meses nos da pie incluso a pensar en realizar nuestros propios proyectos y en enfocarlos desde el principio con otra perspectiva.
