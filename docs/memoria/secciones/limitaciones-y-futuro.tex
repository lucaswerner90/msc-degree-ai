% LIMITACIONES Y PERSPECTIVAS DE FUTURO

\cleardoublepage

\chapter{Limitaciones y\\ Perspectivas de Futuro}
\label{limitaciones-y-futuro}

En este capítulo hablaremos de las limitaciones con las cuales nos encontramos a la hora de desarrollar nuestro trabajo y haremos un análisis de los principales puntos que podrían mejorarse en futuras exploraciones.
\medskip

Comenzaremos hablando de los datos. Por regla general podemos asociar que un buen conjunto de datos marca la diferencia en el aprendizaje de una red neuronal profunda, no solo por su cantidad sino también por la variedad y la veracidad de estos comparados con un entorno real. En cuanto al último punto creo que se tomó la decisión correcta, ya que utilizamos imágenes capturadas por el propio dispositivo con el cual esperábamos usar nuestro agente y por lo tanto la calidad de las imágenes sería la misma. Sin embargo, es posible que nos hayamos quedado cortos en cuanto a la variedad de los escenarios, lo cual hizo que ya desde un principio tuviesemos que pensar que el \textit{overfitting} podría llegar a ser un problema. Esto podría ser facilmente solucionable si procesamos imágenes de diferentes entornos, de manera que nos aseguramos que el agente aprende realmente sobre el problema y no el entorno.
\medskip

Otro punto que supuso una limitación importante fue la limitada capacidad computacional con la que entrenamos nuestros agentes. En un principio se hicieron pruebas utilizando Google Colab, pero la limitación en cuanto a tiempo y el \textit{workflow} al cual nos obligaba a recurrir, hizo que lo descartasemos. También valoramos el uso de una máquina con procesadores gráficos compatibles con la librería \textit{PyTorch}, pero su costo operativo también hizo que no fuese una opción viable.
\medskip

También tenemos que hablar de nuestra falta de experiencia en el desarrollo de estos algoritmos y en general en la rama del aprendizaje por refuerzo, lo cual hizo que sumadas a las dificultades técnicas tuviesemos también que repasar los conceptos aprendidos en multitud de ocasiones.
\medskip

Pensando en trabajos futuros en esta línea de trabajo, dejamos abierta la posibilidad de usar algoritmos que nos ofrezcan una mejor convergencia en los resultados. Un ejemplo de esto podría ser la implementación de los algoritmos \textit{A2C} y \textit{A3C} \citep{DBLP:journals/corr/MnihBMGLHSK16}, que nos ofrecen la posibilidad de realizar múltiples entrenamientos al mismo tiempo o incluso podríamos pensar en la combinación de diferentes técnicas de inteligencia artificial como podrían ser el aprendizaje no supervisado o incluso la utilización de algoritmos genéticos para ayudarnos a encontrar una estructura de red óptima de nuestro problema \citep{cai2018efficient}. 