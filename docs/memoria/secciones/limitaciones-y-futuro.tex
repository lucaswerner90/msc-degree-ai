% LIMITACIONES Y PERSPECTIVAS DE FUTURO

\cleardoublepage

\chapter{Limitaciones y\\ Perspectivas de Futuro}
\label{limitaciones-y-futuro}

En este capítulo hablaremos de las limitaciones con las cuales nos encontramos a la hora de desarrollar nuestro trabajo y haremos un análisis de los principales puntos que podrían mejorarse en futuras exploraciones.
\medskip

Comenzaremos hablando de los datos. Por regla general podemos asociar que un buen conjunto de datos marca la diferencia en el aprendizaje de una red neuronal profunda, no solo por su cantidad sino también por la variedad y la veracidad de estos comparados con un entorno real. En cuanto al último punto creo que se tomó la decisión correcta, ya que utilizamos imágenes capturadas por el propio dispositivo con el cual esperábamos usar nuestro agente y por lo tanto la calidad de las imágenes sería la misma. Sin embargo, es posible que nos hayamos quedado cortos en cuanto a la variedad de los escenarios, lo cual hizo que ya desde un principio tuviesemos que pensar que el \textit{overfitting} podría llegar a ser un problema.
\medskip

Otro punto que supuso una limitación importante fue la limitada capacidad computacional con la que entrenamos nuestros agentes. En un principio se hicieron pruebas utilizando Google Colab, pero la limitación en cuanto a tiempo y el \textit{workflow} al cual nos obligaba a recurrir, hizo que lo descartasemos. También valoramos el uso de una máquina con procesadores gráficos compatibles con la librería \textit{PyTorch}, pero su costo operativo también hizo que no fuese una opción viable.
\medskip